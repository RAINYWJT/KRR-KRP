% This document is compiled using pdfLaTeX
% You can switch XeLaTeX/pdfLaTeX/LaTeX/LuaLaTeX in Settings

\documentclass{article}
\usepackage[utf8]{inputenc}

\title{Assignment 1}
\author{221300079 Juntong Wang}
\date{\today}

\begin{document}

	\maketitle

	\section{Question 1}

	\subsection{(1)}

	Timo is a cow.

	\subsection{(2)}

	Fido is not a dog.

	\subsection{(3)}

	Owing a cow is not enough to recognize a Person as a LivestockOwner.

	\subsection{(4)}

	0.(Zero)

	\section{Question 2}

	\subsection{(1)}
	
	I will translate it into SHOIQ form:(this form is from wiki)

	\begin{enumerate}
		\item[(1)] ML is an AI course taught by ZZH, a professor working at NJU
		
		inclusions:$\{ML\} \sqsubseteq AIcourse \sqcap (\exists teach^-.\{ZZH\})$,$\{ZZH\} \sqsubseteq Professor \sqcap (\exists workAt.\{NJU\})$ 

		\item[(2)] NJU is a university whose members are a school or a department
		
		inclusions:$\{NJU\} \sqsubseteq University \sqcap (\forall hasMember.(School \sqcup Department))$
		
		\item[(3)] NJU has at least 30,000 students

		inclusions:$\{NJU\} \sqsubseteq (\geq 30000 has.Student)$

		\item[(4)] All members of AI School are undergraduates, graduates, or teachers
		
		inclusions:$(\exists memberOf.\{AI School\}) \sqsubseteq (Undergraduate \sqcup Graduate \sqcup Teacher)$
		
		\item[(5)] The domain of the relation ``citizenOf'' consists of countries
		
		inclusions:$(\exists citizenOf.\{\top\}) \sqsubseteq (Country)$
	
	\end{enumerate}

	\subsection{(2)}

		\textbf{Sentence 1:All members of AI School are undergraduates,graduates,or teachers}

		$\forall x (memberOf(x, AI School) \to Undergraduate(x) \lor Graduate(x) \lor Teacher(X))$

		\textbf{Sentence 2:The domain of the relation "citizenOf" consists of countries}

		$\forall x (\exists y citizenOf(x,y) \to Country(x))$

	\section{Question 3}
 
	All the answer is follow the question:

	\begin{itemize}
		\item There is an ontology that has only finite models.
		
		Disprove.\\
		Here is a way to create infinite models.Firstly we assume we have an model:$I = \{\Delta^I, .^I\}$
		And then we assume that:$\{a\} \subseteq \Delta^I$, then we create a new element called:$\{a'\}$,so we could have a new model:
		$I' = \{\Delta^{I'}, .^{I'}\}$,and then we get:$\Delta^{I'} = (\Delta^{I} / a) \cup \{a'\}$\\
		So if we create the new model in this form, we could create infinite models.

		\item Every ontology has either no model or infinite many models.
		
		Prove:\\
		If the ontology has no model, it must in this way:$\top \sqsubseteq \bot$,and from question 2.1 we could get that the number of models is infinite.
		So it has proved.

		\item A satisfiable class must always have a non-empty interpretation.
		
		Prove:\\
		From definition $2.14$, the satisfiability said that:C is satisfiable with respect to $\tau$ iff $C^I \neq \emptyset$ for some model $I$ of $\tau$,
		the satisfiable model must have an interpretation, so proved. 
		
		\item An unsatisfiable class may have a non-empty interpretation in some models.
		
		Disprove:\\
		If the unsatisfiable class have a non-empty interpretation in some models, from definition $2.14$, this is also satisfy the definition of 
		satisfiable class, so it's contradictory.\\

		\item An unsatisfiable class will be a subclass of any other class.
	
		Prove:\\
		From question 2.4 we get that the unsatisfiable class is an emptyset.So an emptyset is always the subclass of any other class.
	
	\end{itemize}


	\section{Question 4}

	All the answer is follow the questions:

	\begin{enumerate}
		\item $\exists r.(A\sqcup B)$:$\{d,f\}$
		\item $\exists s.\exists s.\neg A$:$\{d,e\}$
		\item $\neg A\sqcap\neg B$:$\{f,h,i\}$
		\item $\forall r.(A\sqcup B)$:$\{d,f,g,h,i\}$
		\item $\leq{1}{s}.\top$:$\{e,f,g,h,i\}$
	\end{enumerate}

	\section{Question 5}

	\subsection{(1)}

	All the answer is follow the questions:
	
	\begin{itemize}
		\item $(Q~\sqcap\geq{2}{r}.P)^{\mathcal{I}}$:$\emptyset$
		\item $(\forall r.Q)^{\mathcal{I}}$:$\{b,c,d,e\}$
		\item $(\neg\exists r.Q)^{\mathcal{I}}$:$\{b,c,e\}$
		\item $(\forall r.\top\sqcap\exists r^{-}.P)^{\mathcal{I}}$:$\{b,d,e\}$
		\item $(\exists r^{-}.\bot)^{\mathcal{I}}$:$\emptyset$
	\end{itemize}

	\subsection{(2)}
	
	All the answer is follow the questions:

	\begin{itemize}
		\item $\mathcal{I}\models A\equiv\exists r.B$:True
		\item $\mathcal{I}\models A\sqcap B\sqsubseteq\top$:True
		\item $\mathcal{I}\models \exists r.A\sqsubseteq A\sqcap B$:True
		\item $\mathcal{I}\models \top\sqsubseteq B$:False
		\item $\mathcal{I}\models B\sqsubseteq\exists r.A$:False
	\end{itemize}


	

	\section{Question 6}

	All the answer is follow the questions:

	\begin{itemize}
		\item if $C\sqsubseteq D$ holds, then $\exists r.C\sqsubseteq\exists r.D$ holds.
		
		For this question the proof as follow:\\
		$(\exists r.C)^I$ = $\{d \in \Delta^I | \exists e \in \Delta^I :(d,e) \in r^I and \ e \in C^I\}$\\
		$\sqsubseteq$ $\{d \in \Delta^I | \exists e \in \Delta^I :(d,e) \in r^I and \ e \in C^I\}$ $\sqcup$  $\{d \in \Delta^I | \exists e \in \Delta^I :(d,e) \in r^I and \ e \in D^I / C^I\}$ \\
		= $\{d \in \Delta^I | \exists e \in \Delta^I :(d,e) \in r^I and \ e \in D^I\}$\\
		= $(\exists r.D)^I$

		\item $\exists r.C$ is equivalent to $\leq{1}{r}.\top$.
		
		Disprove as follow:\\
		If we have model like: $r^I = \{(a,b)\},\ C^I = \{a\},\ \Delta^I = \{a,b\} $\\
		So we get $(\exists r.C)^I$ is empty, but the $\leq{1}{r}.\top$ is $\{a,b\}$\\
		So $\exists r.C$ is not equivalent to $\leq{1}{r}.\top$
		
		\item $\leq{0}{r}.\top$ is equivalent to $\forall r.\bot$.
		
		For this question the proof as follow:\\
		$(\leq{0}{r}.\top)^I$ = $\{d \in \Delta^I | \{e \in \Delta^I :(d,e) \in r^I and \ e \in T\} \leq 0 \}$\\
		= $\{d \in \Delta^I | \{e \in \Delta^I :(d,e) \in r^I and \ e \in T\} = \emptyset \}$\\
		= $\{d \in \Delta^I | \{e \in \Delta^I :(d,e) \in r^I \} = \emptyset \}$\\
		= $\{d \in \Delta^I | there \ is \ no \ relation \ (d,e) \in r^I \}$\\

		$(\forall r.\bot)^I$ = $\{d \in \Delta^I | for \ all \ e \in \Delta: (d,e) \in r^I \rightarrow e \in \bot \}$\\
		= $\{d \in \Delta^I | for \ all \ e \in \Delta: (d,e) \in r^I \rightarrow e \in \emptyset \}$\\
		= $\{d \in \Delta^I | there \ is \ no \ relation \ (d,e) \in r^I \}$\\

		\item $\forall r.(A\sqcup B)$ is equivalent to $(\forall r.A)\sqcup(\forall r.B)$.
		
		Disprove as follow:\\
		If we have model:$\Delta^I = \{a,b,c\} , A^I = \{a\}, B^I = \{b\}, r^I = \{(c,a),(c,b)\}$\\
		As the interpretation of $(\forall r.(A\sqcup B))^I$ goes: $\{d \in \Delta^I | for \ all \ e \in \Delta: (d,e) \in r^I \rightarrow e \in (A \cup B)\}$ = $\{a,b,c\}$\\
		But if we get the interpretation of $(\forall r.A)^I\sqcup(\forall r.B)^I$ we get the answer goes:$\{a,b\}$\\
		So $\forall r.(A\sqcup B)$ is not equivalent to $(\forall r.A)\sqcup(\forall r.B)$

		\item $\exists r.(A\sqcup B)$ is equivalent to $(\exists r.A)\sqcup(\exists r.B)$.
		
		For this question the proof as follow:\\
		Firstly:we prove $\exists r.(A\sqcup B)$ $\sqsubseteq $ $(\exists r.A)\sqcup(\exists r.B)$\\
		As the interpretation goes:  $\exists r.(A\sqcup B)$ = $\{d \in \Delta^I | there \ is \ e \in \Delta^I:\ (d,e) \in r^I \ and \ e \in (A \cup B) \}$\\
		$\subseteq$ $\{d \in \Delta^I | there \ is \ e \in \Delta^I:\ (d,e) \in r^I \ and \ e \in A\}$ $\cup$ $\{d \in \Delta^I | there \ is \ e \in \Delta^I:\ (d,e) \in r^I \ and \ e \in B  \}$\\
		= $\{d \in \Delta^I | there \ is \ e \in \Delta^I:\ (d,e) \in r^I \ and \ e \in A or \in B \}$\\

		Secondly:we prove:$(\exists r.A)\sqcup(\exists r.B)$ $\sqsubseteq $ $\exists r.(A\sqcup B)$\\
		As the interpretation goes:$\{d \in \Delta^I | there \ is \ e \in \Delta^I:\ (d,e) \in r^I \ and \ e \in A\}$ $\cup$ $\{d \in \Delta^I | there \ is \ e \in \Delta^I:\ (d,e) \in r^I \ and \ e \in B  \}$\\
		$\subseteq$ $\{d \in \Delta^I | there \ is \ e \in \Delta^I:\ (d,e) \in r^I \ and \ e \in A or \in B \}$\\

		So this question has proved.

	\end{itemize}

	\section{Question 7}

	Here is the proof:
	
	We assume a model as follow:$\Delta^I = \{a,b,c\}$,$Person^I = \{a,b,c\}$,$Parent^I = \{a,b\}$,$Mother^I = \{a\}$,and relationship:$hasChild^I = \{(a,c),(b,c)\}$.
	
	Firstly we could get that:$(\exists\textsf{hasChild}.\textsf{Person})^I$ equals to $\{a,b\}$, and $Person^I$ also equals to $\{a,b\}$, so we get:$\textsf{Parent}\sqsubseteq\exists\textsf{hasChild}.\textsf{Person}$
	
	Secondly we get that:$Mother^I = \{a\}$,so $ \textsf{Mother}\sqsubseteq\textsf{Parent}$.
	
	At this time, we have proved $\mathcal{I}\models\mathcal{T}$.
	
	Then,$Parent^I = \{a,b\}$ which is definitely not belongs to $Mother^I = \{a\}$.

	So we have proved $\mathcal{I}\not\models\textsf{Parent}\sqsubseteq\textsf{Mother}$.


	\section{Question 8}

	Let $\mathcal{T}$ be an $\mathcal{ALC}$ TBox, which is a finite set of concept inclusions. Let $X$ and $Y$ be complex $\mathcal{ALC}$ concepts (note that a complex concept can also be an atomic concept). Show that:
	\begin{itemize}
    	\item $X\sqsubseteq_{\mathcal{T}}Y$ if and only if $X\sqcap\neg Y$ is not satisfiable with respect to $\mathcal{T}$.
    	
		Prove:\\
		$\Rightarrow$: 
		If  $X\sqsubseteq_{\mathcal{T}}Y$,so $X^I \subseteq Y^I$, so $X^I \cap \neg Y^I$ is $emptyset$, so there is no model, so is not satisfiable with respect to $\mathcal{T}$\\
		$\Leftarrow$:
		If $X\sqcap\neg Y$,so $X^I \cap \neg Y^I$ is $emptyset$ is satisfiable,so $X^I \subseteq Y^I$,so $X\sqsubseteq_{\mathcal{T}}Y$.

		\item $X$ is satisfiable with respect to $\mathcal{T}$ if and only if $X\not\sqsubseteq\bot$.

		Prove:\\
		$\Rightarrow$: 
		If $X$ is satisfiable with respect to $\mathcal{T}$,from definition 2.14 we get $X^I \neq \emptyset$, so $X\not\sqsubseteq\bot$.\\
		$\Leftarrow$:
		If $X\not\sqsubseteq\bot$, this means there exists an interpretation or model to satisfy X, so from definition X is satisfiable with respect to $\tau$
		 

	\end{itemize}

	

	\section{Question 9}

	\section{Question 10}

\end{document}
