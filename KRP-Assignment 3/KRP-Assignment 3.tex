\documentclass[12pt,a4paper]{article}
 
                               
\usepackage[T1]{fontenc} % So we can use pretty T1 fonts
\usepackage{libertine}
 
\usepackage[margin=0.5in]{geometry}
\usepackage{amsmath,amssymb,amsfonts}
\usepackage{graphicx}
\usepackage{textcomp}
\usepackage{soul}
\usepackage{hyperref}
\usepackage{xcolor}
\usepackage{caption}
\usepackage{booktabs}
\usepackage{cancel}
\usepackage{algorithm}
\usepackage{algorithmic}
%\usepackage{helvet}
%\usepackage{courier}
\usepackage{mathtools}
\usepackage{pifont}
\usepackage{dashbox}
\usepackage{xspace}
\usepackage{color}
\usepackage{multirow}
\usepackage{url}
%\usepackage{extsizes}
% the following package is optional:
\usepackage{latexsym}
%\usepackage{mathptmx}
\usepackage{stmaryrd}
\usepackage{enumitem}
\newtheorem{definition}{Definition}
\setlength\parindent{0pt}


\newcommand{\alc}{$\mathcal{ALC}$\xspace}
\newcommand{\el}{$\mathcal{EL}$\xspace}
%para nao ficar o retangulo em volta dos links, apenas muda a cor dos caracteres
\hypersetup{ colorlinks,
linkcolor=blue,
filecolor=blue,
urlcolor=blue,
citecolor=blue }

% altera a fonte nas legendas das figuras
%\usepackage[font=small,format=plain,labelfont=bf,up,textfont=times]{caption}
 

\newenvironment{problem}[2][{\color{red}Question}]{\begin{trivlist}
\item[\hskip \labelsep {\bfseries #1}\hskip \labelsep {\bfseries #2.}]}{\end{trivlist}}

\newenvironment{problems}[2][{\color{purple}Question}]{\begin{trivlist}
\item[\hskip \labelsep {\bfseries #1}\hskip \labelsep {\bfseries #2.}]}{\end{trivlist}}

\newenvironment{problemss}[2][{\color{red}\xcancel{Question 10}}]{\begin{trivlist}
\item[\hskip \labelsep {\bfseries #1}\labelsep {\bfseries #2.}]}{\end{trivlist}}

\newenvironment{solution}[2][{\color{blue}Model Solution}]{\begin{trivlist}
\item[\hskip\labelsep {\bfseries #1}\hskip \labelsep {\bfseries #2.}]}{\end{trivlist}}

\setlength{\parskip}{0,3em}  %altera o espaco entre dois paragrafos
\renewcommand{\baselinestretch}{1.1} %altera o espacamento entre as linhas
 
\begin{document}

\title{{\color{blue}KRP --- Assignment 3}}
\author{Instructor: Yizheng Zhao}

 
\maketitle

\textbf{$\star$~\textcolor{gray}{This assignment, due on \underline{\textcolor{blue}{12th May at 23:59}}, contributes to 10\% of the final marks for this course. Please be advised that only Questions 1 --- 10 are mandatory. Nevertheless, students can earn up to one bonus mark by completing Question~11. This bonus mark can potentially augment a student's overall marks but is subject to a maximum total of 100 for the course. By providing bonus marks, we aim to incentivize students to excel in their studies and reward those with a remarkable grasp of the course materials.}}

\begin{problem}{{\color{red}1}}
\textbf{Basic Tableau Algorithm}
\begin{itemize}
    \item Apply the Tableau algorithm $\textsf{consistent}(\mathcal{A})$ to the following ABox:
    \begin{center}
    $\mathcal{A}=\{(b,a):r, (a,b):r, (a,c):s, (c,b):s, a:\exists s.A, b:\forall r.((\forall s.\neg A)\sqcup(\exists r.B)), c:\forall s.(B\sqcap(\forall s.\bot))\}$.
    \end{center}If $\mathcal{A}$ is consistent, draw the model generated by the algorithm.
\end{itemize}
\end{problem}


\begin{problem}{{\color{red}2}}
\textbf{Modification of Tableau Algorithm}\\
We consider an $\mathcal{ALC}$ TBox $\mathcal{T}$ consisting only of the following two kinds of axioms:
\begin{itemize}
    \item  role inclusions of the form $r\sqsubseteq s$, and
    \item role disjointness constraints of the form $\textsf{disjoint}(r,s)$.
\end{itemize}where $r$ and $s$ are role names. An interpretation $\mathcal{I}$ satisfies these axioms if
\begin{itemize}
    \item  $r^{\mathcal{I}}\subseteq s^{\mathcal{I}}$, and
    \item $r^{\mathcal{I}}\cap s^{\mathcal{I}}=\emptyset$, respectively.
\end{itemize}
Modify the Tableau algorithm $\textsf{consistent}(\mathcal{A})$ to decide the consistency of $(\mathcal{T},\mathcal{A})$, where $\mathcal{A}$ is an ABox and $\mathcal{T}$ a TBox containing only role inclusions and role disjointness constraints. Show that the algorithm remains terminating, sound, and complete.
\end{problem}


\begin{problem}{{\color{red}3}}
\textbf{Negation Normal Norm (NNF)}\\
Let $\mathcal{T}$ be an acyclic TBox in NNF. $\mathcal{T}^{\sqsubseteq}$ is obtained from $\mathcal{T}$ by replacing each concept definition $A\equiv C$ with the concept inclusion $A\sqsubseteq C$.

\begin{itemize}
\item[-] Prove that every concept name is satisfiable w.r.t.\ $\mathcal{T}$ iff it is satisfiable w.r.t.\ $\mathcal{T}^{\sqsubseteq}$. Does this also hold for the acyclic TBox $\{A\equiv C\sqcap\neg B, B\equiv P, C\equiv P\}$?
\end{itemize}

\end{problem}


\begin{problem}{{\color{red}4}}
\textbf{Termination}\\
Let $E$ be an \alc-concept. By $\#E$ we denote the number of occurrences of the constructors $\sqcap$, $\sqcup$, $\exists$, $\forall$ in $E$. The multiset $M(E)$ contains, for each occurrence of a subconcept of the form $\neg F$ in $E$, the number $\#F$.

\begin{itemize}
\item[-] Following this representation, prove that exhaustively applying the transformations below to an \alc concept always terminates, regardless of the order of rule application:
\begin{align*}
\neg(E\sqcap F)&\leadsto\neg\neg\neg E\sqcup\neg\neg\neg F\\
\neg(E\sqcup F)&\leadsto\neg\neg\neg E\sqcap\neg\neg\neg F\\
\neg\neg E&\leadsto E\\
\neg(\exists r.E)&\leadsto\forall r.\neg E\\
\neg(\forall r.E)&\leadsto\exists r.\neg E
\end{align*}
\end{itemize}
\end{problem}


\begin{problem}{{\color{red}5}}
\textbf{Tableau Algorithm for ABoxes with Acyclic TBoxes}\\
We consider the Tableau algorithm $\textsf{consistent}(\mathcal{T},\mathcal{A})$ for acyclic TBoxes $\mathcal{T}$, which is obtained from $\textsf{consistent}(\mathcal{A})$ by adding the $\equiv_{1}$-rule and the $\equiv_{2}$-rule for unfolding $\mathcal{T}$.
\begin{itemize}
\item[-] Prove that $\textsf{consistent}(\mathcal{T},\mathcal{A})$ is a decision procedure for the consistency of $\mathcal{ALC}$-knowledge bases with acyclic TBoxes.
\end{itemize}
\end{problem}


\begin{problem}{{\color{red}6}}
\textbf{Tableau Algorithm for ABoxes with Acyclic TBoxes}
\begin{itemize}
\item[-] Use the Tableau algorithm $\textsf{consistent}(\mathcal{T},\mathcal{A})$ for acyclic TBoxes to determine whether the subsumption
\[\neg(\forall r.A)\sqcap\forall r.C\sqsubseteq_{\mathcal{T}}\forall r.E\]holds w.r.t.\ the acyclic TBox
\[\mathcal{T}=\{C\equiv(\exists r.\neg B)\sqcap\neg A, D\equiv\exists r.B, E\equiv\neg(\exists r.A)\sqcap\exists r.D\}.\]
\end{itemize}
\end{problem}


\begin{problem}{{\color{red}7}}
\textbf{Anywhere Blocking}\\
We consider a different form of blocking, which allows individuals to be blocked by individuals who are not necessarily their ancestors, known as anywhere blocking. This approach employs an individual $a$'s age, denoted as $\textsf{age}(a)$, to determine the blocking relationship, instead of relying on the ancestor relation.

The $\textsf{age}$ of an individual is defined as $0$ for individuals that occur in the input ABox $\mathcal{A}$, while a new individual generated by the $n$th application of the $\exists$-rule is assigned an age of $n$. This approach expands the scope of blocking beyond the ancestor relation, enabling individuals to be blocked based on their age, which could result in more effective blocking in certain situations.

Let $\mathcal{A}^{\prime}$ be an ABox obtained by applying the Tableau rules of $\textsf{consistent}(\mathcal{T}, \mathcal{A})$ for general TBoxes. A tree individual $b$ is anywhere blocked by an individual $a$ in $\mathcal{A}^{\prime}$ if
\begin{itemize}
\item[$\bullet$] $\textsf{con}_{\mathcal{A}^{\prime}}(b)\subseteq\textsf{con}_{\mathcal{A}^{\prime}}(a)$,
\item[$\bullet$] $\textsf{age}(a)<\textsf{age}(b)$, and
\item[$\bullet$] $a$ is not blocked.
\end{itemize}
As before, the descendants of $b$ are then also considered blocked.
\begin{itemize}
\item[-] Prove that the Tableau algorithm with anywhere blocking is a decision procedure for the consistency of $\mathcal{ALC}$-knowledge bases with general TBoxes. 
\end{itemize}
\end{problem}


\begin{problem}{{\color{red}8}}
\textbf{Precompletion of Tableau Algorithm}\\
We consider an $\mathcal{ALC}$-knowledge base $\mathcal{K}=(\mathcal{T},\mathcal{A})$ with $\mathcal{T}$ being a general TBox. A \emph{precompletion} of $\mathcal{K}$ is a clash-free ABox $\mathcal{A}$ obtained from $\mathcal{K}$ by exhaustively applying all expansion rules except the $\exists$-rule.

\begin{itemize}
    \item[-] Prove that $\mathcal{K}$ is consistent if, and only if, there is a precompletion $\mathcal{A}$ of $\mathcal{K}$ such that, for all individual names $a$ occurring in $\mathcal{A}$, the concept description $C^{a}_{\mathcal{A}}\coloneqq\underset{a: C\in\mathcal{A}}{\bigsqcap}C$ is satisfiable w.r.t.\ $\mathcal{T}$.
\end{itemize}
\end{problem}


\begin{problem}{{\color{red}9}}
\textbf{Tableau Algorithm for $\mathcal{ALCN}$}
\begin{itemize}
\item[-] Prove soundness and completeness of the Tableau algorithm for $\mathcal{ALCN}$ discussed in the lecture.
\end{itemize}
\end{problem}


\begin{problem}{{\color{red}10}}
\textbf{Tableau Algorithm for $\mathcal{ALCQ}$}\\
We extend the Tableau algorithm from $\mathcal{ALCN}$ to $\mathcal{ALCQ}$ by modifying the $\geq$-rule and the $\leq$-rule as follows:
\begin{center}
    \includegraphics[width=0.75\columnwidth]{12121.png}
    \end{center}
    \begin{itemize}
\item[-] For the knowledge base
    \[(\{C\sqsubseteq E\}, \{a:{\leq}{1r}.(D\sqcap E), (a, b) : r, b : C\sqcap D, (a, c) : r, c: D\sqcap E, c : \neg C\}),\]determine whether it is consistent, and whether the proposed algorithm detects this.
\end{itemize}
\end{problem}


\begin{problems}{{\color{purple}11 (with 1 bonus mark)}}
\textbf{A Complex in \alc Extensions}\\
The DL $\mathcal{S}$ extends \alc with \emph{transitivity axioms} $\textsf{trans}(r)$ for role names $r\in\textsf{R}$. Their semantics is defined as follows: $\mathcal{I}\models\textsf{trans}(r)$ iff $r^{\mathcal{I}}$ is transitive. Furthermore, an $\mathcal{S}$ knowledge base $\mathcal{K}:=(\mathcal{T}, \mathcal{A}, \mathcal{R})$ consists of an \alc knowledge base $(\mathcal{T}, \mathcal{A})$, and an additional RBox $\mathcal{R}$ of transitivity axioms. Prove the following:
\begin{itemize}
\item[-] For an arbitrary TBox $\mathcal{T}$, the concept $C_{\mathcal{T}}$ is defined as $\underset{C\sqsubseteq D\in\mathcal{T}}{\bigsqcap}\neg C\sqcup D$. Then $\mathcal{T}$ and $\mathcal{T}^{\prime}=\{\top\sqsubseteq C_{\mathcal{T}}\}$ have the same models.
\item[-] Let $\mathcal{K}:=\{\mathcal{T}, \mathcal{A}, \mathcal{R}\}$ be a knowledge base such that, without loss of generality, $\mathcal{T}$ consists of a single GCI $\top\sqsubseteq C_{\mathcal{T}}$, and $C_{\mathcal{T}}$ is in NNF. Define the \alc knowledge base $\mathcal{K}^{+}:=(\mathcal{T}^{+}, \mathcal{A})$ where
\begin{align*}
    &\mathcal{T}^{+}:=\mathcal{T}\cup\{\forall r.C\sqsubseteq\forall r.\forall r.C~|~\textsf{trans}(r)\in\mathcal{R}~\text{and}~\forall r.C\in\textsf{Sub}(C_{\mathcal{T}})\}.
\end{align*}
Then $\mathcal{K}$ is consistent, if and only if, $\mathcal{K}^{+}$ is consistent. Consequently, the Tableau algorithm for \alc can also be used for $\mathcal{S}$.
\end{itemize}
\end{problems}

\end{document}
